\section{Custom scene graph nodes} \label{sec:customSGNodes}

For our project, we implemented a bunch of custom scene graph nodes.
The source of all of them is placed in the directory code/nodes/ .

\subsection*{Skybox node} \label{subsec:skyboxNode}
For the sky in the background, a separate \js{SkyboxSGNode} was introduced.
The node works by rendering a 2x2x2 cube around the camera, ranging from $-1$ to 1 in each direction x, y, z.
While rendering, all depth checks are disabled, effectively making the skybox draw over anything rendered before it on the screen.
This is the reason why the cube does not need to bee bigger than the whole scene.
Additionally, depth writing is disabled, meaning that any object rendered will overdraw the skybox, making it appear in the background every time.

\subsection*{Particle system} \label{subsec:particleSystemNode}
The particle system is implemented as a node on its own, enabling it to be transformed, rotated and scaled as any other object.
More details on the implementation and the technical details particle system are in chapter~\ref{sec:particleSystem}.

\subsection*{Texture node} \label{subsec:textureNode}
For applying textures, we took the TextureSGNode from the labs, and adapted.
Each texture node increments a variable in the context determining the texture unit which should be used.
If no one is set in the shader, texture unit 0 is used by default (\footnote{provide code sample?})

\subsection*{Animation node} \label{subsec:animationNode}
For realizing the animations, a special AnimationSGNode was introduced.
It is basically a TransformationSGNode, but instead of using a static transformation matrix,
it is supplied a function, which generates a arbitrary transformation matrix based on the
current animation time \footnote{Satz k\"urzen}.

\subsection*{Animation trigger node} \label{subsec:animationTriggerNode}
A AnimationTriggerSGNode enables the definition of such-called \say{trigger point}.
A trigger point starts an animation as soon as the camera enters a defined radius around the origin of the trigger.

As long as the animation is not triggered yet, all children of this node receive a time of 0 through the context,
effectively halting the animation of all AnimationSGNode s and other time using nodes.
When the AnimationTriggerSGNode is triggered, the current time point is stored.

A AnimationTriggerSGNode can only be triggered once and can not be reset for now.

\subsection*{Spotlight node} \label{subsec:spotlightNode}
The spotlight node is used to define a spotlight in the world.
It inherits from the LightSGNode and provides two additional uniforms:
\begin{description}
	\item[angle] Specifies the angle of the cone in which the light illuminates
	\item[direction] Defines the direction of the Spotlight. This is a direction vector
\end{description}